\documentclass[12pt]{article}

\usepackage{fullpage}
\usepackage{graphicx, rotating, booktabs} 
\usepackage{times} 
\usepackage{natbib} 
\usepackage{indentfirst} 
\usepackage{setspace}
\usepackage{grffile} 
\usepackage{hyperref}
\usepackage{adjustbox}
\setcitestyle{aysep{}}


\singlespace
\title{\textbf{Summary of Previous Empirical Work on the Arms-Alliances Tradeoff}}
\author{Joshua Alley\footnote{Graduate Student,
Department of Political Science, Texas A\&M University.}}
\date{{\normalsize \today}}

\bibliographystyle{apsr}

\begin{document}

\maketitle 


\section*{Introduction}

What do we know about the association between arms and alliances? Domestic military spending and international alliances are two of the most important security policies states have. While most scholars examine arms or alliances in isolation, others address how states mix these two policies in pursuit of security. 

To assess the state of research on the arms-alliances tradeoff, I undertook a comprehensive a literature review. The review cataloged the primary claim, theory, research design, and evidence of 25 papers.\footnote{In the interest of parsimony and time, I did not comprehensively summarize work on free-riding in NATO, which is a literature unto itself. Instead, I focused on the papers that made major methodological or theoretical innovations.} I was interested in how each study contributed to our understanding of the general relationship between arms and alliances. 

Our knowledge of the arms-alliances tradeoff is incomplete at best. There are two competing theoretical camps--- one sees arms and allies as substitutes, while the other argues that these policies are complements. Theories of arms and alliances as complements do a poor job developing mechanisms to support their claim. Even with these theoretical issues, it is hard to assess the balance of knowledge because of serious empirical problems. 

Few studies generate a credible estimate of the association between arms and alliances. Many scholars test general claims with limited samples and misspecified models. Due to poor empirical design in many studies, it is difficult to identify meaningful patterns in the results. 

The most trustworthy models suggest that any association between arms and alliances is conditional. The best general model of the arms-alliances tradeoff only finds substitution in a particular type of alliances. Mixed results from studies of specific alliances underline this conclusion. 

Current theories of the arms-alliances tradeoff do not make conditional claims, however. Instead, researchers make exclusive claims that arms and allies are substitutes or complements. I start this paper by summarizing the theoretical logic underlying each paradigm. Then, I turn to an assessment of the research designs that tested those theoretical claims. The last section concludes by discussing open issues in the arms-alliances tradeoff literature and considerations for future research. 


\section*{Arms and Allies as Substitutes}





\section*{Arms and Allies as Complements}



\section*{Empirical Challenges}


To the credit of prior researchers, estimating the association between arms and alliances is difficult. The exact observable implications of an arms-alliances are unclear, beyond a sense that marginal changes are most meaningful \citep{Morrow2000, Starr2000}. Studies must account for contextual factors that might drive simultaneous changes in arms and allies, measurement challenges, and scope conditions. National military spending is highly autocorrelated, and the skewed distribution of the variable can present substantial challenges for conventional regression estimators. 

Few studies estimate the general association between arms and alliances. Only seven used military expenditures as the dependent variable, and accounted for the presence of multiple alliances. Four of these seven models are untrustworthy, as they fail to control for key correlates of spending and use limited samples. 

For example, \citet{MorganPalmer2003} estimate a regress changes in military expenditure on indicators of state and alliance power with no control variables in an empirical sample is state-year observations in the year after a state's alliance portfolio changes. Other studies focus only on great power alliances \citep{ConybeareSandler1990, Conybeare1994, Diehl1994, MostSiverson1987}, with little attention to possible differences in how these states make defense allocations. 

The other 17 studies look at dynamics within specific alliances. These studies can address alliance-level variation in much more detail, looking at specific changes in the political context or nature of the alliance itself. Seven of these studies examine NATO or the Warsaw Pact only. Much of the work on specific alliances tests whether larger members have a higher defense burden, in accordance with the primary prediction of the economic theory of alliances. Early studies of free-riding in specific alliances rely on correlation coefficients or regressions with 20 or fewer observations, where there is a high risk of spurious results from omitted variable bias or outliers \citep{OlsonZeckhauser1966, Starr1974, Reisinger1983, Siroky2012}. The focus on specific alliances is often driven by data availability--- NATO members in particular have reliable spending data for the entire post-war period. 

Most studies, whether specific or general, ignore autocorrelation in the data-generating process of military spending. Four of the 24 studies use an estimator or specification that can address autocorrelation in military spending. Four others use some version of a differenced dependent variable. 

The best general model of the arms alliances tradeoff is the work of \citet{DigiuseppePoast2016}, who find that substitution of alliances for arms is conditional on the credibility of the alliance. They argue that that democracies make more credible commitments, and estimate regression models of the association between defense pacts with a democracy and the natural log of military spending, using a lagged dependent variable to account for autocorrelation. All told, the research design in this paper is a marked improvement over prior work, though it offers a crude take on intra-alliance variation.  

Some models of specific alliance dynamics provide more detailed estimates of how membership in a particular treaty affects defense spending. In these studies, researchers can leverage specific knowledge to identify omitted variables and critical contextual factors. \citet{PluemperNeumayer2015}'s quasi-spatial model of how NATO members respond to US and Soviet spending is the best evidence of free-riding by NATO members. \citet{Sorokin1994} takes a similar approach to estimating the responses of four diverse states to the military spending of a key ally with a two-stage least squares GLS estimator. 



\section*{Overall Assessment}

What can we conclude about our understanding of the arms-alliances tradeoff? First, if theory testing depends on the accumulation of evidence, we have accumulated little reliable evidence. What we reliably know about the arms-alliance tradeoff depends on 4 general studies and 3 or 4 alliance-specific studies. 

The detail in specific studies of alliances points to a major challenge for future research--- accounting for the different levels of analysis in the arms-alliances tradeoff and variation across those levels. International organizations such as alliances are a separate level of analysis from states \citep{Mattes2012, Chibaetal2015}. Because the outcome of interest is usually at the state level, scholars have relied on crude summaries of the overall alliance portfolio \citep{Conybeare1992}, or only estimated the effect of a few alliances \citep{OnealWhatley1996}. 



\bibliography{C:/Users/jkalley14/Dropbox/Research/MasterBibliography}  
%\bibliography{C:/Users/Josh/Dropbox/Research/MasterBibliography} 





\end{document}