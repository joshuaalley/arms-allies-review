\documentclass[12pt]{article}

\usepackage{fullpage}
\usepackage{graphicx, rotating, booktabs} 
\usepackage{times} 
\usepackage{natbib} 
\usepackage{indentfirst} 
\usepackage{setspace}
\usepackage{grffile} 
\usepackage{hyperref}
\usepackage{adjustbox}
\setcitestyle{aysep{}}


\singlespace
\title{\textbf{Summary of Previous Empirical Work on the Arms-Alliances Tradeoff}}
\author{Joshua Alley\footnote{Graduate Student,
Department of Political Science, Texas A\&M University.}}
\date{{\normalsize \today}}

\bibliographystyle{apsr}

\begin{document}

\maketitle 


%----------------------------------
\section{Introduction}

% Set up question
How does membership in an international alliance affect a state's military spending? 
There are two competing answers to this question. 
One group expects that alliance membership leads to reductions in spending, while the other predicts that alliance membership increases military expenditures. 


% assessment of the two predictions
The modal perspective is that the two predictions are incompatible, which has stunted theoretical development. 
The predictions themselves are clear, but the models behind these predictions have different strengths and weaknesses. 
Furthermore, our ability to assess each prediction has been complicated by serious shortcomings in research design, as many tests use misspecified statistical models or inappropriate samples.
As a result, there is ample room for theoretical and empirical progress in our understanding of how alliance membership affects military spending. 
 

% General framework and fix terms. 
To understand current models of alliance membership and military spending, we can situate them in a general framework. 
Predictions of increased or decreased military spending from alliance participation rely on concepts from a standard utility maximization model in microeconomics. 
In this model, an actor maximizes their satisfaction from consuming two different goods given their preferences and constraints. 
Preferences over combinations of the two goods are represented by an indifference curve. 
The prices of each good and the actor's budget constrain how much they can consume. 


This model predicts that the actor will spend their whole budget on some combination of the two goods to maximize their utility.
Because the entire budget is spent, changes in the consumption of the two goods are connected, as long as the budget remains the same.
If the price of one good decreases, and consumption of the other decreases, the two goods are substitutes. 
But if one good becomes more affordable and consumption of the other increases, the two goods are complements. 


% General framework applied to arms-alliances tradeoff
In international relations, states consume arms or alliances to increase their security, subject to their resource constraints. 
So states are the actors, while arms and alliances are the goods they consume. 
The prediction that alliance membership and defense spending are negatively correlated implies that the two policies are substitutes--- increased consumption of alliances leads to decreased consumption of arms. 
If alliance participation is positively correlated with military spending, then the two policies are complements, as increased consumption of alliances leads to increased consumption of arms. 


% Summarize set of models. 
There are several different models behind predictions of substitution and complementarity. 
Scholars have used a private goods model of foreign policy substitution and a public goods model to explain substitution of alliances for arms. 
Models of complementarity are less well developed, but the most general relies on an income effect, whereby alliance membership makes military spending more valuable. 
The rest of this paper explains and critiques these models in more detail. 


% Roadmap
The first section of this review summarizes explanations and evidence for the prediction that arms and alliances are substitutes. 
The second section details explanations and evidence for expectations of complementary. 
The third section draws conclusions about our knowledge of the relationship between arms and alliances, and suggests directions for future research. 





%----------------------------------
\section{Arms and Alliances as Substitutes}

% Connecting paragraph
Most academics and popular discourse holds that alliance membership leads to reductions in military spending. 
There are two distinct explanations of the process behind this substitution of alliances for arms. 
The substitution theory of foreign policy rely on the fact that arms and alliances both provide security to explain substitution. 
The public goods theory of alliances, also known as the economic theory of alliances, argues that the security alliances produce is a public good, which generates incentives for members to free-ride on the defense spending of their peers. 


% Common mechanism 
Both models of substitution use the opportunity costs of military spending to explain why alliance members have the incentive to reduce military spending. 
Every dollar spent on the military cannot be spent on other policies \citep{Powell1993, JacksonMorelli2008, Fearon2018}. 
Although military spending can have some positive externalities \citep{WhittenWilliams2011}, it at best has a mixed association with economic growth \citep{DegerSen1995, ShinWard1999, ScanlanJenkins2001, AlptekinLevine2012}. 
For leaders seeking to provide public and private goods to supporters, reductions in military spending increase funding for other policies.\footnote{Military-led regimes are a notable exception to this pattern.} 
Alliance membership provides additional security, which gives leaders the freedom to cut military spending and reallocate those funds to other policies. 


% Why address two models? 
Although they make similar predictions, there are significant differences between the two models, which means they must be addressed separately.
The policy and academic implications of substitution theory of foreign policy are quite different from those of the public goods theory of alliances. 
If the security from an alliance is a public good, then alliances are collective security organizations. 
Because public goods theory predicts free-riding, it has a morally charged implication that alliance members are shirking their duty to maintain collective security. 
But if security is specific to individual states, then reduced defense spending is less morally problematic.
Substitution theory of foreign policy focuses on security for individual states, and I summarize that logic first. 



\subsection{Substitution Theory of Foreign Policy}

% Introduce broad idea
The core insight behind substitution theory of foreign policy is simple--- states can use multiple policies to attain their goals \citep{MostStarr1989}. 
For example, to increase their prosperity, states can adjust their exchange rates, trade policies, or investment regulations. 
The mix of policies states choose depends on the relative efficacy of each policy, the costs of implementation, and their available resources. 


% Application to international security
Security from external threat is one of the most important goals for states. 
In an anarchic international system, states regularly confront the prospect of armed conflict. 
Losing a war creates substantial costs in lost resources and foreign policy freedom.
So security depends on the ability of states to successfully prosecute a war, and military capability is a crucial determinant of success in conflict. 
While success in conflict is not guaranteed by a preponderance of capability \citep{Arreguin-Toft2005, Sullivan2012}, states with more troops and military resources tend to win in international conflict. 
Therefore, states consider the balance of forces before initiating international conflicts, and states can use arms or alliances to shift that balance in their favor. 


% Both alliances and arms provide military capability
Domestic military spending and international alliances both provide military capability, which increases a state's security \citep{Morrow1993}. 
Alliances provide access to the capabilities of other states through promises of support in the event of a conflict. 
Military expenditures can be used to build or import weapons, train soldiers, and provide other defense goods. 


% And here's where substitution happens
States then balance the costs and efficiency of arms and alliances \citep{Morrow1993, Conybeare1994}. 
If forming an alliance provides adequate capability, states can rely on allied capability for security instead to devoting more resources to the military. 
When states gain extra security from an alliance, they can reduce military spending and reallocate those resources to other policies. 
Therefore, substitution theory of foreign policy expects that alliance participation leads to decreased defense expenditures.


% DiGiuseppe and Poast extension: Leave this here or in overall assessment. 
\citet{DigiuseppePoast2016} extend this theory of substitution by noting that alliance credibility may modify the relationship between arms and alliances. 
They argue that states will only reduce military spending if they believe their allies will honor their treaty obligations, and that because democracies make more credible commitments, defense treaties with democracies will produce substitution. 
The overall logic of substitution theory of foreign policy is clear, but what evidence is there for its prediction that alliance membership leads to less military spending? 



\subsubsection{Evidence}


% Initial outline 
There are few valid tests of the substitution theory of foreign policy theory in the case of arms and alliances. Testing predictions from substitution theories of foreign policy is challenging, because specifying the exact behavioral outcomes is difficult \citep{Starr2000}. Most studies have tried to establish an association between alliance membership and military spending, but many of the empirical models are misspecified. 

% Evidence
Case studies of Great Power relations in Europe during the 1860s \citep{Morrow1993} and Egypt during the Cold War \citep{BarnettLevy1991} suggest that states can tradeoff between domestic investment in arms and alliance membership. However, \citet{MostSiverson1987} find no association between allied spending and domestic military spending using regressions and cross-tabulations of Great Power spending in Europe from 1870 to 1914. 
\citet{Sorokin1994} applies a two-stage generalized least squares model to four separate time series of military spending in Austria-Hungary and France from 1880 to 1913 as well as Syria and Israel from 1963 to 1988, and finds that France and Syria responded to increases in allied spending by reducing their own military expenditures. 
Last, \citet{Conybeare1994} finds that great powers with more demanding alliance obligations have lower ratios of military manpower to allied manpower. 
The focus on individual countries and alliances in theses models leaves us with one general empirical model of substitution


% DG&Poast as the best (only) general model 
\citet{DigiuseppePoast2016} use a panel data regression with a lagged DV, panel-specific autocorrelation correction, and robust set of control variables to test whether membership in a defense pact with a democracy leads to decreased military spending. 
They find a robust negative association between defense pacts with a democracy and military spending in a sample of all states in the international system from 1950 to 2001.\footnote{The only remaining threat to inference I can think of is selection on unobservables \citep{Chaudoinetal2016}, which I'm in the process of checking.} 
DiGiusseppe and Poast's results support the idea that states only substitute alliances for arms in particular circumstances.
Therefore, substitution theory of foreign policy has some explanatory power, which is reinforced by tests of a different prediction of the theory. 


% tests of the alliance formation implication
Another implication of substitution theory of foreign policy is that when the costs of domestic arms are high, states will form alliances to increase their security instead. 
Two studies test this prediction using panel data.\footnote{The use of panel data in work on alliance formation is somewhat unusual, and the probit models in these studies do not fit the data well.}
\citet{Kimball2010} finds that high demand for social policies, which she measures using the infant mortality rate, increases the propensity of a states to form an alliances in a given year.
\citet{AllenDigiuseppe2013} argue that limited access to credit will increase the opportunity costs of military spending, and find that states suffering an external debt crisis are also more likely to form alliances. 
Neither theory and research design can rule out the alternative argument that states with serious economic difficulties form alliances to secure economic aid from larger partners. 
These studies provide some tentative evidence for another prediction of the substitution theory of foreign policy, which increases the plausibility of that argument. 


% Assessment and transition
Overall, the logic of substitution theory of foreign policy and associated empirical results suggest that this model contains useful insights about the relationship between alliance membership and military spending. 
Because both arms and alliances provide military capability, and military spending has domestic opportunity costs, the incentive for states to rely on allied military spending instead of their own is clear. 
The public goods theory of alliances also predicts that alliance membership leads to reduced military spending, but uses a different process. 


\subsection{Public Goods Theory of Alliances} 






\subsubsection{Evidence}



\subsection{Assessment of the Substitution Prediction}


% Something on imperfect substitutes


% Research design challenges





%----------------------------------
\section{Arms and Alliances as Complements}



\subsection{Justifications of Complementarity} 





\subsection{Evidence of Complementarity} 





\subsection{Assessment of the Complementarity Prediction} 






%----------------------------------
\section{Conclusion}



\subsection{Overall Development}




\subsection{Future Research}






\section*{Appendix}

\subsection*{Visual Summary of Prior Results} 


\begin{table}[hbt!]
\begin{tabular}{lccc}
     & Substitutes & Complements & Null \\
\hline
\citet{MostSiverson1987} &  &  & X \\
\citet{Morrow1993} & X &  &  \\ 
\citet{Conybeare1994} & X & &  \\
\citet{Diehl1994} &  & X &  \\
\citet{Goldsmith2003} &  &  & X \\
\citet{MorganPalmer2006} &  & X & \\ 
\citet{QuirozFlores2011} &  & X &  \\ 
\citet{DigiuseppePoast2016} & X & & \\ 
\citet{Horowitzetal2017} &  & X & \\ 
\hline
\end{tabular}
\caption{General Findings of Association Between Alliance Membership and Military Spending}
\end{table}


\begin{table}[hbt!]
\begin{tabular}{lcc}
  & Free-Riding & Alliance Scope \\
\hline
\citet{OlsonZeckhauser1966} & X  & NATO 1964 \\
\citet{Starr1974} & X & Warsaw Pact 1967-1971 \\
\citet{Reisinger1983} &  & Warsaw Pact 1970-1978 \\
\citet{Thies1987} & Mixed & Seven pre-1945 Alliances \\ 
\citet{ConybeareSandler1990} &  & Triple Alliance \& Triple Entente 1880-1914 \\
\citet{Palmer1990} &   & NATO 1950-1978 \\
\citet{Chenetal1996} &  & Arab League 1950-1988 \\
\citet{OnealWhatley1996} & X & NATO, Rio Pact \& Arab League 1953-1988 \\
\citet{Siroky2012} &  & Quintuple Alliance Members 1820 \\
\citet{PluemperNeumayer2015} & X & NATO 1956-1988 \\
\hline 
\end{tabular}
\caption{Findings on Free-Riding in Alliances. Most of these studies uses military expenditures as a share of GDP for the dependent variable, and assess whether it is correlated with economic size. A positive correlation between GDP or GNP and the defense burden among alliance members is the standard evidence of free-riding.}
\end{table}






\bibliography{C:/Users/jkalley14/Dropbox/Research/MasterBibliography}  
%\bibliography{C:/Users/Josh/Dropbox/Research/MasterBibliography} 





\end{document}