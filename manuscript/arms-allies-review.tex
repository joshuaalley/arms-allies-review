\documentclass[12pt]{article}

\usepackage{fullpage}
\usepackage{graphicx, rotating, booktabs} 
\usepackage{times} 
\usepackage{natbib} 
\usepackage{indentfirst} 
\usepackage{setspace}
\usepackage{grffile} 
\usepackage{hyperref}
\usepackage{adjustbox}
\setcitestyle{aysep{}}


\singlespace
\title{\textbf{Summary of Previous Empirical Work on the Arms-Alliances Tradeoff}}
\author{Joshua Alley\footnote{Graduate Student,
Department of Political Science, Texas A\&M University.}}
\date{{\normalsize \today}}

\bibliographystyle{apsr}

\begin{document}

\maketitle 


\section*{Introduction}

What do we know about the association between arms and alliances? Domestic military spending and international alliances are two of the most important policies in international security. While most scholars examine arms or alliances in isolation, others address how states mix these two policies in pursuit of security. 

To assess the state of research on the arms-alliances tradeoff, I undertook a comprehensive literature review. The review cataloged the primary claim, theory, research design, and evidence of 26 papers.\footnote{In the interest of parsimony and time, I did not comprehensively summarize work on free-riding in NATO, which is a literature unto itself. Instead, I focused on the papers that made major methodological or theoretical innovations.} I was interested in how each study contributed to our understanding of the general relationship between arms and alliances. 

Our knowledge of the arms-alliances tradeoff is incomplete at best. There are two competing theoretical camps--- one sees arms and allies as substitutes, while the other argues that these policies are complements. Theories of arms and alliances as complements do a poor job developing mechanisms to support their claim. Even with these theoretical issues, it is hard to assess the balance of knowledge because of serious empirical problems. 

Few studies generate a credible estimate of the association between arms and alliances. Many scholars test general claims with limited samples and misspecified models. Due to poor empirical design in many studies, it is difficult to identify meaningful patterns in the results. 

The most trustworthy models suggest that any association between arms and alliances is conditional. The best general model of the arms-alliances tradeoff only finds substitution in a particular type of alliances. Mixed results from studies of specific alliances underline this conclusion. 

Current theories of the arms-alliances tradeoff do not make conditional claims, however. Instead, researchers make exclusive claims that arms and allies are substitutes or complements. I start this paper by summarizing the theoretical logic underlying each paradigm. Then, I assess the research designs that tested those theoretical claims. The last section concludes by discussing open issues in the arms-alliances tradeoff literature and considerations for future research. 


\section*{Arms and Allies as Substitutes}

If arms and alliances are substitutes, then additional alliances or capability from alliance partners should lead to decreases in domestic military spending.\footnote{All empirical tests focus on this prediction, although substitution logic also predicts that when alliances become more costly, states will turn to domestic arms.} Two distinct theories expect this inverse association between alliances and domestic arms. Substitution theories of the arms-alliances tradeoff and the economic theory of alliances focus on how arms and allies both produce security, leading states to replace domestic military spending with allied spending. 

Both theories use the opportunity costs of military spending to motivate substitution. Although military spending can have positive economic and technological spinoffs \citep{DegerSen1995, WhittenWilliams2011}, allocating funds to the military takes resources away from other social and economic policies. Leaders can use these other public and private goods to ensure their political survival \citep{BDMetal2002}. Therefore if states can rely on their allies to provide security, they have ample domestic political motives to reduce spending. 

Predictions of substitution between arms and alliances rely on the insight that states can use multiple policies means to achieve their foreign policy goals \citep{MostStarr1989}. \citet{Morrow1993} argues that domestic arms and alliances are both sources of military capability, allowing states to substitute between the two in their search for security. Instead of internal balancing through spending more on domestic arms, states can rely on external balancing through allied capability \citep{Conybeare1992}. 

The marginal costs of arms and alliances determine which policy states use to increase their security \citep{Sorokin1994}. For instance, some evidence suggests that as the opportunity cost of military spending increases, states are more likely to form alliances \citep{Kimball2010, AllenDigiuseppe2013}. Measuring the foreign policy costs of a potential alliance has proved more difficult, and is the main barrier to examining whether higher costs of an alliance lead states to rely more on military spending. These differences in the type of costs associated with each policy means that arms and alliances are not perfect replacements for one another. 

Arms and allies are imperfect substitutes. States can rely on their own arms in any contingency, but domestic military capabilities take a long time to develop. Because members have divergent foreign policy interests, alliances are a less reliable source of capability than domestic arms, but provide immediate capability gains. The moment a treaty enters into force, alliance members gain their partner's support.

As an alliance becomes more reliable, that treaty becomes more like domestic arms as a source of capability. Therefore, more credible alliance treaties are a better substitute for domestic arms. \citet{DigiuseppePoast2016} expand on this insight by arguing that because defense pacts with a democracy are more credible, these alliances will lead to reductions in spending. 

The theoretical logic of substitution theory is clear, but we have little evidence for these claims. Only four studies test substitution of alliances for arms using military spending or personnel as the dependent variable. This lack of attention reflects more emphasis on possible free riding in alliances. 

The economic theory of alliances also predicts that alliances lead to reduced domestic arms, but relies on a different mechanism. \citet{OlsonZeckhauser1966} treat security from an alliance as a public good, which generates a collective action problem for members. Each members military spending is a contribution to the public good, but all members have incentives to free ride on the effort of their partners. As a result, they predict that larger members of an alliance will bear a higher defense burden, because these states have a higher absolute value of the public good. Olson and Zeckhauser interpret a positive correlation between GDP and the ratio of defense spending to GDP as evidence of free-riding. Under this theory, alliance membership leads small states to reduce defense spending. 

Olson and Zeckhauser's claims sparked extensive debate about free riding by alliance members. Much of the discussion focused on NATO, but other scholars checked the correlation between defense burdens and GDP in other alliances, with mixed results \citep{Reisinger1983, Thies1987, GatesTerasawa1992, OnealWhatley1996, Siroky2012}. The best evidence of free-riding is \citet{PluemperNeumayer2015}, who show that many NATO allies are unresponsive or even reduce spending when growth in Soviet spending exceeds growth in US spending, and that the degree of free-riding depends on distance from the USSR. As with substitution theories, free-riding may only occur when treaty commitments are sufficiently credible \citep{GatesTerasawa1992}. 

A theoretical extension of the economic theory of alliances conceptualizes military spending as having public and private benefits \citep{ConybeareSandler1990}. These joint product models lead to different predictions, because as the ratio of private to public benefits increases, we should observe less free-riding \citep{Murdoch1995, SandlerHartley2001}. Other studies divided types of military spending according to the benefit they provide, especially nuclear and non-nuclear defense effort \citep{Hansenetal1990}. Nuclear arms provide a pure public good through deterrence, while conventional spending has private benefits such as increased border security. 

Emphasizing free riding creates several issues for the economic theory of alliances. First, the concept of free-riding itself is problematic. Reductions in spending by NATO members could reflect decreased threat perceptions or a bargain with the United States \citep{Lanoszka2015}. Distinguishing between free-riding and exchange is difficult, especially if smaller alliance members trade security for autonomy in asymmetric alliances \citep{Morrow1991}. 

Emphasizing free riding limits the applicability of evidence from the economic theory of alliances to a general understanding of the arms-alliances tradeoff. The standard dependent variable is the defense burden--- military spending as a share of GDP. It is difficult to identify a model of this ratio dependent variable, casting serious doubt on many of the studies that rely on correlations or rank orderings of defense burdens and economic size for evidence \citep{PluemperNeumayer2015}. 

Despite problems from the focus on free-riding in the economic theory of alliances, substitution theories of alliances have a clear internal logic and mechanism. This theoretical clarity allowed scholars to set general expectations about the relationship between arms and alliances--- substitution theory is the best known theoretical paradigm in the arms-alliances tradeoff literature. Policymakers complaints about free-riding further reinforced the perception that alliances lead to reduced defense effort by junior partners. While popular, substitution theory is incomplete.

The major weakness of substitution theories is their emphasis on security as the only output of an alliance. Substitution theory acknowledges that multiple policies can advance a particular goal, but it does not acknowledge multiple foreign policy goals. Therefore, this perspective cannot capture the full range of how alliances may affect domestic arms. 

States have a range of foreign policy goals. Even security focused paradigms like neorealism admit states may have different preferences over whether to maintain or change the status quo \citep{Schweller1994, Walt2009}. Substitution theory addresses some of the domestic political incentives states face in making security policy, but it provides little detail about possible differences in foreign policy. By contrast, theories of arms and alliances as complements have a less clear internal logic, but consider diverse foreign policy aims. 


\section*{Arms and Allies as Complements}

The literature on arms and alliances as complements is less developed. These theories expect that arms and alliances are positively correlated--- states that form an alliance will increase military spending. Scholars have proposed several explanations for this prediction.

\citet{Diehl1994} argues that arms and alliances are not comparable policies and have multiple purposes besides deterrence, so substitution is unlikely. He then lists several mechanisms that might generate complementarity between arms and alliances, but is not clear about which is most important. These mechanisms include the possibility that alliances require increased foreign policy activity, or that major powers increase military spending to cover free-riding by junior alliance partners. The most general argument asserts that autonomy benefits from an alliance increase the marginal value of military spending.

\citet{MorganPalmer2006} also use an increase in the value of military spending to motivate complementarity in their study of how states use different foreign policy instruments. Their theory starts with the premise that states have two foreign policy goals--- maintenance and change. States then allocate all their foreign policy resources to pursue a mix of maintenance and change across a range of issues. Due to budget constraints, there is a tradeoff between maintenance and change. To examine substitution between different foreign policy instruments, they start with the assumption that states choose the most efficient policy tool to realize their goals. 

Given this assumption, there are three reasons states might increase their use of a particular policy. The first is an increase the overall resources available to the state, and the second is changes in relative preferences for maintenance and change. The last path is changes in the efficiency of different policies, which is the only path that leads to substitution. In this model, substitution is motivated by states moving resources from less to more efficient policies, which they argue is uncommon. 

Rather, Morgan and Palmer argue that alliances increase the amount of resources available to the state, which frees up resources that can then be dedicated to pursuing other goals. Alliances expand the foreign policy possibilities for a state, which makes them more likely to increase military spending. As an illustrative example, they argue that an alliance with the US gave Britain and France extra resources to pursue change in the Middle East during the 1957 Suez crisis.  

By only examining foreign policy tradeoffs, Morgan and Palmer ignore domestic incentives to reduce military spending. Because states do not expend the same kinds of resources on domestic arms and alliances, the concept of a foreign policy ``budget'' is nebulous. These weaknesses aside, Morgan and Palmer make a valuable contribution by distinguishing between goals of maintenance and change, and possible tradeoffs between the two.

Neither Diehl nor Morgan and Palmer adequately explain the process by which joining an alliance increases military spending. But if arms and allies are imperfect substitutes as Diehl argues, then these two policies also have some characteristics that allow for substitution. Greater clarity about the mechanism by which alliances increase the marginal benefits of military spending is necessary. 

Another mechanism behind complementarity is cooperation and policy coordination between alliance members. \citet{Palmer1990} conceptualizes relations among alliance members as an iterated prisoner's dilemma, where states can communicate and use side issues to overcome collective action problems. \citet{QuirozFlores2011} attributes a positive spatial correlation in the defense burden of allied states to increased cooperation between allies.

The cooperation thesis acknowledges collective action problems in the economic theory of alliances, but argues that these problems can be overcome. There is some evidence that alliances facilitate cooperation in other issue areas \citep{Gowa1995, GowaMansfield2004, Poast2012, Poast2013}. Cooperation on side issues will not necessarily lead to increases in spending as states overcome collective action problems. Small states can use concessions in other issue areas as part of a bargain with their patron that allows for reduced military spending. Intra-alliance cooperation is an insufficient explanation of complementarity between arms and allies. 

The challenge for complementarity theory is clear--- scholars must develop better theoretical explanations for their prediction that alliance membership leads to increases in military spending. Neither general arguments or the cooperation thesis provide a sufficient explanation for possible complementarity. The shortcomings of current theories do not imply that scholars should rule out complementarity between arms and alliances, however. Theories of arms and allies as complements still make a valuable contribution to knowledge of the arms alliances tradeoff, by considering diverse foreign policy goals. 


\section*{The State of Arms-Allies Theory}

Given arguments that arms and alliances are complements or substitutes, what is the overall theoretical development of the literature? Work on arms and alliances is characterized by two disjointed perspectives with distinct strengths and weaknesses. Debate over whether arms and allies are substitutes or complements treats the two perspectives as incompatible. Framing substitutes and complements as competing hypotheses has stunted theoretical progress, as neither perspective incorporates the key insight of the other. 

Substitution theories have not engaged with the claim that states have diverse foreign policy goals. Complementarity theories do not consider domestic political incentives for security-seeking states to rely on allied capability. A general theory of the arms-alliances tradeoff must incorporate both of these elements--- domestic political incentives to substitute and diverse foreign policy goals. 

Because the strengths and weaknesses of complementarity and substitution theories mirror each other, it may be possible to combine the two. Rather than pitting the two perspectives against each other, a more fruitful approach should consider whether each applies in different circumstances. This combination implies a conditional association between arms and alliances. 

\citet{DigiuseppePoast2016}'s argument that substitution hinges on the credibility of an alliance treaty is a useful first step. By acknowledging heterogeneous effects of alliances, the theory and results suggest that arms and alliances are not uniformly substitutes or complements. The main shortcoming of this work is that while it acknowledges findings of a positive correlation between arms and alliances, it does not address those cases. 

DiGiuseppe and Poast employed two theoretical mechanisms from democratic credibility and substitution theory. Explaining when arms and allies are complements will require more effort. But complementarity and substitution theories share a common element that can facilitate theoretical progress. 

Both substitution and complementarity theories use an income effect to connect a state's alliance portfolio to changes in military spending. Income effects address how an actor changes its consumption basket after an increase in its budget.\footnote{For example, rising income usually leads individuals to add more protein to their diet.} Both theories predict that gains from an alliance will change how states utilize other resources. The difference between the two paradigms is that they focus on changes in the consumption of different goods. Substitution theory focuses on shifts in domestic consumption, while complementarity theories consider changes in foreign policy. 

Therefore, the key task for scholars is articulating when states will emphasize domestic or international gains. Whether an alliance produces security or autonomy \citep{Morrow1991}, is essential to understanding how states use the capability gains from a treaty. Security reflects interests in maintaining the status quo, while autonomy is the ability of states to seek changes in the status quo. Security-producing alliances will facilitate domestic gains, while autonomy-generating alliances will lead to foreign policy gains. 

A degree of security is a necessity for all states, as it allows them to set other policies as they see fit \citep{Lake1996}. Under substitution, additional allied capability causes a state to shift its domestic consumption away from providing security towards other economic and social goods that are otherwise luxuries. In complementarity theories, an alliance expands what a state can achieve in its foreign policy, so states can consume additional foreign policy goods. 

Contextual factors may also alter the nature of the arms-alliances tradeoff. Between nuclear weapons, changing norms about international conquest \citep{Fazal2011}, economic interdependence \citep{Frieden2006}, and changes in the cost/complexity of weapons systems \citep{Bitzinger2003} the international environment has changed in dramatic ways, especially after 1945. So far, no theories of the arms-alliances tradeoff have addressed whether these shifts in the international environment alter states' goals. Constraints on how much change or autonomy a state can achieve in foreign policy may lead to more emphasis on domestic gains from alliances.   

Regardless of how theories of the arms-allies tradeoff evolve, clear hypotheses are indispensable. Without precise theoretical predictions, the risk of spurious results from poor research designs increases. Most current theories do not specify all the possible observable implications, or do not clarify the appropriate dependent variable. This lack of clarity is reflect in poor research designs. Another issue with predictions from current theories is their inadequate attention to possible scope conditions. 

The most salient example of a problem with scope conditions is the debate over free-riding in alliances. Most scholars ignore possible differences between major and non-major powers. The economic theory of alliances expects greater free riding by smaller alliance members, but many tests of this theory examine alliances between major powers \citep{Thies1987, ConybeareSandler1990, Siroky2012}. 

If smaller states are more likely to use alliances as a source of security, rather than to increase their foreign policy autonomy \citep{Morrow1991}, then autonomy-seeking major powers have less incentive to substitute alliances for arms. Other theories beyond the economic theory of alliances must consider these differences between major and minor powers. If the relationship between arms and allies is fundamentally different for major powers, studies of major power alliances will only tell part of the story.  


The theoretical and empirical importance of scope conditions points to the other major problem in scholarship on the arms-alliances tradeoff--- many early empirical results have limited viability. Absent reliable empirical results, assessing the relative value of different theories is impossible. Despite the general claims in different theories, most studies of the arms-allies tradeoff provide poor evidence for their claims. Many researchers use limited samples of states or alliances, or fail to account for critical empirical issues.   


\section*{Empirical Challenges}

To the credit of prior researchers, many of whom were working with limited methodological tools, estimating the association between arms and alliances is difficult. The exact observable implications of an arms-alliances tradeoff are unclear, beyond a sense that marginal changes are most meaningful \citep{Morrow2000, Starr2000}. Any research design must consider contextual factors that might drive simultaneous changes in arms and allies, measurement challenges, and scope conditions. The standard dependent variable is also challenging. National military spending is highly autocorrelated, and the skewed distribution of the variable can present substantial challenges for conventional regression estimators. 

Few studies estimate the general association between arms and alliances. Only seven used military expenditures as the dependent variable, and accounted for the presence of multiple alliances. Four of these seven models are untrustworthy, as they fail to control for key correlates of spending and use limited samples. 

For example, \citet{MorganPalmer2003} regress changes in military expenditure on indicators of state and alliance capabilities in a sample of the years after a state's alliance portfolio changed. Other studies focus only on great power alliances \citep{ConybeareSandler1990, Conybeare1994, Diehl1994, MostSiverson1987}, with little attention to possible differences in how these states make defense allocations. While great powers face salient security threats, a better empirical approach is to control for differences in threat. 

The other 17 studies look at dynamics within specific alliances. These studies can address alliance-level variation in much more detail, looking at specific changes in the political context or nature of the alliance itself. Seven of these studies examine NATO or the Warsaw Pact only. Much of the work on specific alliances tests whether larger members have a higher defense burden, in accordance with the primary prediction of the economic theory of alliances. Early studies of free-riding in specific alliances rely on correlation coefficients or regressions with 20 or fewer observations \citep{OlsonZeckhauser1966, Starr1974, Reisinger1983, Siroky2012}, where there is a high risk of spurious results from omitted variable bias or outliers. The focus on specific alliances is often driven by data availability--- NATO members and great powers have more data on military spending. 

Most studies, whether specific or general, ignore autocorrelation in the data-generating process of military spending. Four of the 24 studies use an estimator or specification that can address autocorrelation in military spending. Four others use a differenced dependent variable. 

The best general model of the arms alliances tradeoff is the work of \citet{DigiuseppePoast2016}, who find that substitution of alliances for arms is conditional on the credibility of the alliance. They argue that that democracies make more credible commitments, and estimate regression models of the association between defense pacts with a democracy and the natural log of military spending, using a lagged dependent variable to account for autocorrelation. All told, the research design in this paper is a marked improvement over prior work, though it offers a crude take on intra-alliance variation.  

Some models of specific alliance dynamics provide more detailed estimates of how membership in a particular treaty affects defense spending. In these studies, researchers can leverage specific knowledge to identify omitted variables and critical contextual factors. \citet{PluemperNeumayer2015}'s quasi-spatial model of how NATO members respond to US and Soviet spending is the best evidence of free-riding by NATO members. \citet{Sorokin1994} takes a similar approach to estimating the responses of four diverse states to the military spending of a key ally with a two-stage least squares GLS estimator. 



\section*{Overall Assessment}

What can we conclude about our understanding of the arms-alliances tradeoff? First, if theory testing depends on the accumulation of evidence, we have accumulated little reliable evidence. What we reliably know about the arms-alliance tradeoff depends on 4 general studies and 3 or 4 alliance-specific studies. 

The detail in specific studies of alliances points to a major challenge for future research--- accounting for the different levels of analysis in the arms-alliances tradeoff and variation across those levels. International organizations such as alliances are a separate level of analysis from states \citep{Mattes2012, Chibaetal2015}. Because the outcome of interest is usually at the state level, scholars have relied on crude summaries of the overall alliance portfolio \citep{Conybeare1992}, or only estimated the effect of a few alliances \citep{OnealWhatley1996}. 

% Neither set of theories does a good job thinking about the general international context. 

% Role of alliance spending/operationalization of membership

% Can also consider implications of costs in forming alliances for arms- no one has really done that yet, and it is the inverse of the substitution logic. 



%\bibliography{C:/Users/jkalley14/Dropbox/Research/MasterBibliography}  
\bibliography{C:/Users/Josh/Dropbox/Research/MasterBibliography} 





\end{document}