\documentclass[12pt]{article}

\usepackage{fullpage}
\usepackage{graphicx, rotating, booktabs} 
\usepackage{times} 
\usepackage{natbib} 
\usepackage{indentfirst} 
\usepackage{setspace}
\usepackage{grffile} 
\usepackage{hyperref}
\usepackage{adjustbox}
\setcitestyle{aysep{}}


\singlespace
\title{\textbf{Summary of Previous Empirical Work on the Arms-Alliances Tradeoff}}
\author{Joshua Alley\footnote{Graduate Student,
Department of Political Science, Texas A\&M University.}}
\date{{\normalsize \today}}

\bibliographystyle{apsr}

\begin{document}

\maketitle 


%----------------------------------
\section{Introduction}

% Set up question
How does membership in an international alliance affect a state's military spending? 
There are two competing answers to this question. 
One group expects that alliance membership leads to reductions in spending, while the other predicts that alliance membership increases military expenditures. 


% assessment of the two predictions
The modal perspective is that the two predictions are incompatible, which has stunted theoretical development. 
The predictions themselves are clear, but the models behind these predictions have different strengths and weaknesses. 
Furthermore, our ability to assess each prediction has been complicated by serious shortcomings in research design, as many tests use misspecified statistical models or inappropriate samples.
As a result, there is ample room for theoretical and empirical progress in our understanding of how alliance membership affects military spending. 
 

% General framework and fix terms. 
To understand current models of alliance membership and military spending, we can situate them in a general framework. 
Predictions of increased or decreased military spending from alliance participation rely on concepts from a standard utility maximization model in microeconomics. 
In this model, an actor maximizes their satisfaction from consuming two different goods given their preferences and constraints. 
Preferences over combinations of the two goods are represented by an indifference curve. 
The prices of each good and the actor's budget constrain how much they can consume. 


This model predicts that the actor will spend their whole budget on some combination of the two goods to maximize their utility.
Because the entire budget is spent, changes in the consumption of the two goods are connected, as long as the budget remains the same.
If the price of one good decreases, and consumption of the other decreases, the two goods are substitutes. 
But if one good becomes more affordable and consumption of the other increases, the two goods are complements. 


% General framework applied to arms-alliances tradeoff
In international relations, states consume arms or alliances to increase their security, subject to their resource constraints. 
So states are the actors, while arms and alliances are the goods they consume. 
The prediction that alliance membership and defense spending are negatively correlated implies that the two policies are substitutes--- increased consumption of alliances leads to decreased consumption of arms. 
If alliance participation is positively correlated with military spending, then the two policies are complements, as increased consumption of alliances leads to increased consumption of arms. 


% Summarize set of models. 
There are several different models behind predictions of substitution and complementarity. 
Scholars have used a private goods model of foreign policy substitution and a public goods model to explain substitution of alliances for arms. 
Models of complementarity are less well developed, but the most general relies on an income effect, whereby alliance membership makes military spending more valuable. 
The rest of this paper explains and critiques these models in more detail. 


% Roadmap
The first section of this review summarizes explanations and evidence for the prediction that arms and alliances are substitutes. 
The second section details explanations and evidence for expectations of complementary. 
The third section draws conclusions about our knowledge of the relationship between arms and alliances, and suggests directions for future research. 





%----------------------------------
\section{Arms and Alliances as Substitutes}



\subsection{Substitution Theory of Foreign Policy}




\subsection{Public Goods Theory of Alliances} 




\subsection{Assessment of the Substitution Prediction}







%----------------------------------
\section{Arms and Alliances as Complements}



\subsection{Justifications of Complementarity} 





\subsection{Evidence of Complementarity} 





\subsection{Assessment of the Complementarity Prediction} 






%----------------------------------
\section{Conclusion}



\subsection{Overall Development}




\subsection{Future Research}






\section*{Appendix}

\subsection*{Visual Summary of Prior Results} 


\begin{table}[hbt!]
\begin{tabular}{lccc}
     & Substitutes & Complements & Null \\
\hline
\citet{MostSiverson1987} &  &  & X \\
\citet{Morrow1993} & X &  &  \\ 
\citet{Conybeare1994} & X & &  \\
\citet{Diehl1994} &  & X &  \\
\citet{Goldsmith2003} &  &  & X \\
\citet{MorganPalmer2006} &  & X & \\ 
\citet{QuirozFlores2011} &  & X &  \\ 
\citet{DigiuseppePoast2016} & X & & \\ 
\citet{Horowitzetal2017} &  & X & \\ 
\hline
\end{tabular}
\caption{General Findings of Association Between Alliance Membership and Military Spending}
\end{table}


\begin{table}[hbt!]
\begin{tabular}{lcc}
  & Free-Riding & Alliance Scope \\
\hline
\citet{OlsonZeckhauser1966} & X  & NATO 1964 \\
\citet{Starr1974} & X & Warsaw Pact 1967-1971 \\
\citet{Reisinger1983} &  & Warsaw Pact 1970-1978 \\
\citet{Thies1987} & Mixed & Seven pre-1945 Alliances \\ 
\citet{ConybeareSandler1990} &  & Triple Alliance \& Triple Entente 1880-1914 \\
\citet{Palmer1990} &   & NATO 1950-1978 \\
\citet{Chenetal1996} &  & Arab League 1950-1988 \\
\citet{OnealWhatley1996} & X & NATO, Rio Pact \& Arab League 1953-1988 \\
\citet{Siroky2012} &  & Quintuple Alliance Members 1820 \\
\citet{PluemperNeumayer2015} & X & NATO 1956-1988 \\
\hline 
\end{tabular}
\caption{Findings on Free-Riding in Alliances. Most of these studies uses military expenditures as a share of GDP for the dependent variable, and assess whether it is correlated with economic size. A positive correlation between GDP or GNP and the defense burden among alliance members is the standard evidence of free-riding.}
\end{table}






\bibliography{C:/Users/jkalley14/Dropbox/Research/MasterBibliography}  
%\bibliography{C:/Users/Josh/Dropbox/Research/MasterBibliography} 





\end{document}