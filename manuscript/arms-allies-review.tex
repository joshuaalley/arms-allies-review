\documentclass[12pt]{article}

\usepackage{fullpage}
\usepackage{graphicx, rotating, booktabs} 
\usepackage{times} 
\usepackage{natbib} 
\usepackage{indentfirst} 
\usepackage{setspace}
\usepackage{grffile} 
\usepackage{hyperref}
\usepackage{adjustbox}
\setcitestyle{aysep{}}


\singlespace
\title{\textbf{Summary of Previous Empirical Work on the Arms-Alliances Tradeoff}}
\author{Joshua Alley\footnote{Graduate Student,
Department of Political Science, Texas A\&M University.}}
\date{{\normalsize \today}}

\bibliographystyle{apsr}

\begin{document}

\maketitle 


\section*{Introduction}

What do we know about the association between arms and alliances? Domestic military spending and international alliances are two of the most important policies in international security. While most scholars examine arms or alliances in isolation, others address how states mix these two policies in pursuit of security. 

To assess the state of research on the arms-alliances tradeoff, I undertook a comprehensive a literature review. The review cataloged the primary claim, theory, research design, and evidence of 26 papers.\footnote{In the interest of parsimony and time, I did not comprehensively summarize work on free-riding in NATO, which is a literature unto itself. Instead, I focused on the papers that made major methodological or theoretical innovations.} I was interested in how each study contributed to our understanding of the general relationship between arms and alliances. 

Our knowledge of the arms-alliances tradeoff is incomplete at best. There are two competing theoretical camps--- one sees arms and allies as substitutes, while the other argues that these policies are complements. Theories of arms and alliances as complements do a poor job developing mechanisms to support their claim. Even with these theoretical issues, it is hard to assess the balance of knowledge because of serious empirical problems. 

Few studies generate a credible estimate of the association between arms and alliances. Many scholars test general claims with limited samples and misspecified models. Due to poor empirical design in many studies, it is difficult to identify meaningful patterns in the results. 

The most trustworthy models suggest that any association between arms and alliances is conditional. The best general model of the arms-alliances tradeoff only finds substitution in a particular type of alliances. Mixed results from studies of specific alliances underline this conclusion. 

Current theories of the arms-alliances tradeoff do not make conditional claims, however. Instead, researchers make exclusive claims that arms and allies are substitutes or complements. I start this paper by summarizing the theoretical logic underlying each paradigm. Then, I turn to an assessment of the research designs that tested those theoretical claims. The last section concludes by discussing open issues in the arms-alliances tradeoff literature and considerations for future research. 


\section*{Arms and Allies as Substitutes}

If arms and alliances are substitutes, then additional alliances or capability from alliance partners should lead to decreases in domestic arms.\footnote{All empirical tests focus on this prediction, although the substitution logic also predicts that when alliances become more costly, states will turn to domestic arms.} Two distinct theories predict this inverse association between alliances and domestic arms. Substitution theories of the arms-alliances tradeoff and the economic theory of alliances focus on how arms and allies both produce security, allowing states to replace domestic military spending with allied spending. 

Both theories rely on the opportunity costs of military spending. Although military spending can have positive economic and technological spinoffs \citep{DegerSen1995, WhittenWilliams2011}, allocating resources to the military takes those resources away from other social and economic policies. If states can rely on their allies to provide security, they have ample domestic political motives to reduce spending. 

Predictions of substitution between arms and alliances rely on the insight that states have multiple means to achieve their desired ends in foreign policy \citep{MostStarr1989}. \citet{Morrow1993} argues that domestic arms and alliances are both sources of military capability, allowing states to substitute between the two in their search for security. Instead of internal balancing through spending more on domestic arms, states can rely on external balancing through allied capability \citep{Conybeare1992}. 

Therefore, differences in the marginal costs of arms and alliances determines which policy states use to increase their security \citep{Sorokin1994}. For instance, some evidence suggests that as the opportunity cost of military spending increases, states are more likely to form alliances \citep{Kimball2010, AllenDigiuseppe2013}. Measuring the foreign policy costs of a potential alliance has proved more difficult, and is the main barrier to examining whether higher costs of an alliance lead states to rely more on military spending. Due to their distinct characteristics, arms and allies have different costs. 

Arms and allies are imperfect substitutes. States can rely on their own arms in any contingency, but domestic military capabilities take a long time to develop. Because members have divergent foreign policy interests, alliances are a less reliable source of capability than domestic arms, but provide immediate capability gains. The moment a treaty enters into force, alliance members have access to their partner's support.

As an alliance treaty becomes more reliable, it becomes more like domestic arms as a source of capability. Therefore, more credible alliance treaties are a better substitute for domestic arms. \citet{DigiuseppePoast2016} expand on this insight by arguing that because defense pacts with a democracy are more credible, these alliances will lead to reductions in spending. 

The theoretical logic of substitution theory is clear, but few studies test these claims. Only four studies test substitution of alliances for arms using military spending or personnel as the dependent variable. This lack of attention is the result of scholars spending more time discussing free riding in alliances. 

The economic theory of alliances makes a similar prediction to substitution theories, but relies on a different mechanism. \citet{OlsonZeckhauser1966} treat the security from an alliance as a public good, which generates a collective action problem for members. Each members military spending is a contribution to the public good. As a result, they predict that larger members of an alliance will bear a higher defense burden, because these states have a higher absolute value of the public good. Olson and Zeckhauser interpret a positive correlation between GDP and the ratio of defense spending to GDP as evidence of free-riding. In this model, alliance membership leads small states to reduce defense spending. 

Olson and Zeckhauser's claims sparked extensive debate about the extent of free riding by alliance members. Much of the discussion focused on NATO, but other scholars checked for similar correlations in other alliances, with mixed results \citep{Reisinger1983, Thies1987, GatesTerasawa1992, OnealWhatley1996, Siroky2012}. The best evidence of free-riding is \citet{PluemperNeumayer2015}, who show that many NATO allies are unresponsive or even reduce spending when growth in Soviet spending exceeds growth in US spending, and that the degree of free-riding depends on distance from the USSR. As with substitution theories, free-riding may only occur when treaty commitments are sufficiently credible, but reduced credibility also reduces the deterrent value of an alliance \citep{GatesTerasawa1992}. 

A fruitful theoretical extension of the economic theory of alliances conceptualizes military spending as having public and private benefits \citep{ConybeareSandler1990}. These joint product models lead to different predictions, because as the ratio of private to public benefits increases, we should observe less free-riding \citep{Murdoch1995, SandlerHartley2001}. Other studies divided types of military spending according to the benefit they provide, especially nuclear and non-nuclear defense effort \citep{Hansenetal1990}. Nuclear arms provide the pure public good of deterrence, while conventional spending has private benefits such as increasing border security. 

The concept of free-riding itself is problematic. Reductions in spending by NATO members could reflect decreased threat perceptions or a bargain with the United States \citep{Lanoszka2015}. Distinguishing between free-riding and exchange is difficult, especially if smaller alliance members trade security for autonomy in asymmetric alliances \citep{Morrow1991}. 

Issues with the concept of free riding restrict the applicability of evidence from those studies to a general understanding of the arms-alliances tradeoff. Because free-riding focuses on disproportionate military spending by alliance members, the standard dependent variable is defense spending as a share of GDP. It is difficult to identify a model of this ratio dependent variable, casting serious doubt on many of the studies that rely on correlations or rank orderings of defense burdens and economic size for evidence \citep{PluemperNeumayer2015}. 

Overall, substitution theories of alliances have a clear internal logic and mechanism. This theoretical clarity and ubiquitous claims of free riding framed general expectations about the relationship between arms and alliances. Even so, these theories have an important deficiency. 

The major theoretical weakness of substitution theories is their emphasis on security as the only output of an alliance. Security from external aggression is only one of several alliance purposes. Substitution theory does not capture the full range of how alliances may affect domestic arms. By contrast, theories of arms and alliances as complements have a less clear internal logic, but allow for multiple purposes for an alliance. 



\section*{Arms and Allies as Complements}





\section*{Empirical Challenges}


To the credit of prior researchers, estimating the association between arms and alliances is difficult. The exact observable implications of an arms-alliances are unclear, beyond a sense that marginal changes are most meaningful \citep{Morrow2000, Starr2000}. Studies must account for contextual factors that might drive simultaneous changes in arms and allies, measurement challenges, and scope conditions. National military spending is highly autocorrelated, and the skewed distribution of the variable can present substantial challenges for conventional regression estimators. 

Few studies estimate the general association between arms and alliances. Only seven used military expenditures as the dependent variable, and accounted for the presence of multiple alliances. Four of these seven models are untrustworthy, as they fail to control for key correlates of spending and use limited samples. 

For example, \citet{MorganPalmer2003} regress changes in military expenditure on indicators of state and alliance capabilities with no control variables in a sample of the years after a state's alliance portfolio changed. Other studies focus only on great power alliances \citep{ConybeareSandler1990, Conybeare1994, Diehl1994, MostSiverson1987}, with little attention to possible differences in how these states make defense allocations. 

The other 17 studies look at dynamics within specific alliances. These studies can address alliance-level variation in much more detail, looking at specific changes in the political context or nature of the alliance itself. Seven of these studies examine NATO or the Warsaw Pact only. Much of the work on specific alliances tests whether larger members have a higher defense burden, in accordance with the primary prediction of the economic theory of alliances. Early studies of free-riding in specific alliances rely on correlation coefficients or regressions with 20 or fewer observations, where there is a high risk of spurious results from omitted variable bias or outliers \citep{OlsonZeckhauser1966, Starr1974, Reisinger1983, Siroky2012}. The focus on specific alliances is often driven by data availability--- NATO members in particular have reliable spending data for the entire post-war period. 

Most studies, whether specific or general, ignore autocorrelation in the data-generating process of military spending. Four of the 24 studies use an estimator or specification that can address autocorrelation in military spending. Four others use some version of a differenced dependent variable. 

The best general model of the arms alliances tradeoff is the work of \citet{DigiuseppePoast2016}, who find that substitution of alliances for arms is conditional on the credibility of the alliance. They argue that that democracies make more credible commitments, and estimate regression models of the association between defense pacts with a democracy and the natural log of military spending, using a lagged dependent variable to account for autocorrelation. All told, the research design in this paper is a marked improvement over prior work, though it offers a crude take on intra-alliance variation.  

Some models of specific alliance dynamics provide more detailed estimates of how membership in a particular treaty affects defense spending. In these studies, researchers can leverage specific knowledge to identify omitted variables and critical contextual factors. \citet{PluemperNeumayer2015}'s quasi-spatial model of how NATO members respond to US and Soviet spending is the best evidence of free-riding by NATO members. \citet{Sorokin1994} takes a similar approach to estimating the responses of four diverse states to the military spending of a key ally with a two-stage least squares GLS estimator. 



\section*{Overall Assessment}

What can we conclude about our understanding of the arms-alliances tradeoff? First, if theory testing depends on the accumulation of evidence, we have accumulated little reliable evidence. What we reliably know about the arms-alliance tradeoff depends on 4 general studies and 3 or 4 alliance-specific studies. 

The detail in specific studies of alliances points to a major challenge for future research--- accounting for the different levels of analysis in the arms-alliances tradeoff and variation across those levels. International organizations such as alliances are a separate level of analysis from states \citep{Mattes2012, Chibaetal2015}. Because the outcome of interest is usually at the state level, scholars have relied on crude summaries of the overall alliance portfolio \citep{Conybeare1992}, or only estimated the effect of a few alliances \citep{OnealWhatley1996}. 

% Neither set of theories does a good job thinking about the general international context. 

% Minimal attention to other alliances 

% Can also consider implications of costs in forming alliances for arms- no one has really done that yet, and it is the inverse of the substitution logic. 



\bibliography{C:/Users/jkalley14/Dropbox/Research/MasterBibliography}  
%\bibliography{C:/Users/Josh/Dropbox/Research/MasterBibliography} 





\end{document}