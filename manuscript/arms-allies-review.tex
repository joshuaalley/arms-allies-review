\documentclass[12pt]{article}

\usepackage{fullpage}
\usepackage{graphicx, rotating, booktabs} 
\usepackage{times} 
\usepackage{natbib} 
\usepackage{indentfirst} 
\usepackage{setspace}
\usepackage{grffile} 
\usepackage{hyperref}
\usepackage{adjustbox}
\setcitestyle{aysep{}}


\singlespace
\title{\textbf{Review Essay on the Arms-Alliances Tradeoff}}
\author{Joshua Alley\footnote{Graduate Student,
Department of Political Science, Texas A\&M University.}}
\date{{\normalsize \today}}

\bibliographystyle{apsr}

\begin{document}

\maketitle 

\doublespace



%----------------------------------
\section{Introduction}

% Set up question
How does membership in an international alliance affect a state's military spending? 
There are two competing answers to this question. 
One group expects that alliance membership leads to reductions in spending, while the other predicts that alliance membership increases military expenditures. 


% assessment of the two predictions
Most scholars treat these two predictions as incompatible, which has stunted theoretical development. 
Models of increases or decreases in spending by alliance members have different strengths and weaknesses. 
Some models focus on domestic political incentives to reduce military spending, while other emphasize foreign policy gains from alliances. 


Furthermore, our ability to assess each prediction has been complicated by serious shortcomings in research design.
Many empirical tests use misspecified statistical models or inappropriate samples.
As a result, there is ample room for theoretical and empirical progress in our understanding of how alliance membership affects military spending. 
 

% Summarize set of models. 
There are several models behind these different predictions. 
Scholars have used a private goods model of foreign policy substitution and a public goods model to predict a negative association between alliance participation and military spending. 
There are five different which expect alliance participation will increase defense expenditures. 
The rest of this paper explains and critiques these models in more detail. 


% Roadmap
The first section of this review summarizes explanations and evidence of a negative relationship. 
The second section details explanations and evidence for a positive association between alliance and arms. 
The final section identifies gaps in our knowledge of the relationship between arms and alliances, and suggests future research questions to fill in those gaps. 





%----------------------------------
\section{Why Alliance Participation Reduces Military Spending}

% Connecting paragraph
Most academics and popular discourse holds that alliance membership leads to reductions in military spending. 
There are two distinct explanations of the process behind reduced military spending by alliance members. 
The substitution theory of foreign policy relies on the fact that arms and alliances both provide security. 
The public goods theory of alliances, also known as the economic theory of alliances, argues security from an alliances is a public good, which creates incentives for members to free-ride on the defense spending of their peers. 


% Common mechanism 
Both models use the opportunity costs of military spending to explain why alliance members want to reduce military spending. 
Every dollar spent on the military cannot be spent on other goods \citep{Powell1993, JacksonMorelli2008, Fearon2018}. 
Although military spending can have some positive externalities \citep{WhittenWilliams2011}, it at best has a mixed association with economic growth \citep{DegerSen1995, ShinWard1999, ScanlanJenkins2001, AlptekinLevine2012}. 
By reducing military spending, leaders increase funding for other policies and provide public or private goods to remain in power \citep{BDMetal2002}.\footnote{Military-led regimes are a notable exception to this pattern.} 
Alliance membership provides additional security, which gives leaders the ability to reduce military spending and reallocate those funds to other policies. 


% Why address two models? 
Although they make similar predictions, the policy and academic implications of substitution theory of foreign policy are distinct from the public goods theory of alliances. 
If the security from an alliance is a public good, then alliances are collective security organizations. 
Because public goods theory predicts free-riding, it has the morally charged implication that alliance members are shirking their duty to maintain collective security. 
But if security is specific to individual states, then reduced defense spending is less morally problematic.
Substitution theory of foreign policy focuses on individual states, and I summarize that logic first. 



\subsection{Substitution Theory of Foreign Policy}

% Introduce broad idea
The core idea behind substitution theory of foreign policy is simple--- states can use multiple policies to attain their goals \citep{MostStarr1989}. 
For example, to increase their wealth, states can adjust their exchange rates, trade policies, or investment regulations. 
The mix of policies states choose depends on the relative efficacy of each policy, the costs of implementation, and their available resources. 


% Application to international security
Security from external threat is one of the most important goals for states. 
In an anarchic international system, states regularly confront the prospect of armed conflict. 
Losing a war creates substantial costs in lost resources and foreign policy freedom.

Because security depends on war-fighting ability, military capability is a crucial determinant of success in conflict. 
While success in conflict is not guaranteed by a preponderance of capability \citep{Arreguin-Toft2005, Sullivan2012}, states with more troops and military resources tend to win in international conflict \citep{Fearon2018}. 
Therefore, states consider the balance of forces before initiating international conflicts, and arms or alliances change that balance. 


% Both alliances and arms provide military capability
Domestic military spending and international alliances provide military capability, which increases a state's security \citep{Morrow1993}. 
Alliances access to the capabilities of other states through promises of support in the event of a conflict. 
Military expenditures can build or import weapons, train soldiers, and provide other defense goods. 


% And here's where substitution happens
States then balance the costs and efficiency of arms and alliances to produce security \citep{Morrow1993, Conybeare1994}. 
If forming an alliance provides adequate capability, states can rely on allied capability for security instead of devoting more resources to the military. 
Extra security from an alliance allows states to reduce military spending and reallocate those resources to other policies. 
Therefore, substitution theory of foreign policy expects that alliance participation leads to decreased defense expenditures.


% DiGiuseppe and Poast extension: Leave this here or in overall assessment. 
\citet{DigiuseppePoast2016} extend this theory by arguing that alliance credibility modifies the relationship between arms and alliances. 
States will only reduce military spending if they believe their allies will honor their treaty obligations.
Because democracies make more credible commitments than other states, defense treaties with democracies lead to reduced military spending. 
The overall logic of substitution theory of foreign policy is coherent, but what evidence is there for its prediction that alliance membership leads to less military spending? 



\subsubsection{Evidence}


% Initial outline 
There are few tests of how substitution theory of foreign policy theory applies to arms and alliances. 
Testing predictions from these theories is challenging \citep{Starr2000}. 
Studies of the association between alliance membership and military spending employ many techniques. 


% Evidence
Case studies of Great Power relations in Europe during the 1860s \citep{Morrow1993} and Egypt during the Cold War \citep{BarnettLevy1991} suggest that states can tradeoff between domestic investment in arms and alliance membership. However, \citet{MostSiverson1987} find no association between allied spending and domestic military spending using regressions and cross-tabulations of Great Power spending in Europe from 1870 to 1914. 
\citet{Sorokin1994} applies a two-stage generalized least squares model to individual time series of military spending in Austria-Hungary and France from 1880 to 1913 as well as Syria and Israel from 1963 to 1988, and finds that France and Syria responded to increases in allied spending by reducing their own military expenditures. 
Last, \citet{Conybeare1994} finds that great powers with more demanding alliance obligations have lower ratios of military manpower to allied manpower. 
All of these paper focus on a few countries or alliances. 


% DG&Poast as the best (only) general model 
There is only one general test of the foreign policy theory of substitution. 
\citet{DigiuseppePoast2016} use a panel data regression with a lagged DV, panel-specific autocorrelation correction, and robust set of control variables to test whether participation in a defense pact with a democracy leads to decreased military spending. 
They find a robust negative association between defense pacts with a democracy and military spending in a sample of all states in the international system from 1950 to 2001.\footnote{Their comparison has a problem with selection on unobservables \citep{Chaudoinetal2016}, where the kinds of states that form defense pacts with democracies have other characteristics that also drive reductions in defense spending.} 
DiGiuseppe and Poast's results support their prediction that states only substitute alliances for arms in particular circumstances.
These and other results suggest that substitution theory of foreign policy has some explanatory power.


% tests of the alliance formation implication
Tests of a different prediction from the substitution theory of foreign policy reinforce its value. 
Another implication of substitution theory of foreign policy is that when military spending has high opportunity costs, states will form alliances to increase their security. 
Two studies test this prediction using panel data.\footnote{The use of panel data in work on alliance formation is somewhat unusual, and the probit models in these studies do not fit the data well.}
\citet{Kimball2010} finds that high demand for social policies, which she measures using the infant mortality rate, increases the propensity of states to form an alliance.
\citet{AllenDigiuseppe2013} argue that limited access to credit increases the opportunity costs of military spending, and find that states suffering an external debt crisis are more likely to form alliances. 
%Neither theory and research design can rule out the alternative argument that states with serious economic difficulties form alliances to secure economic aid from larger partners. 


% Assessment and transition
Overall, the logic of substitution theory of foreign policy and associated empirical results suggest that this model explains part of the relationship between alliance membership and military spending. 
Because both arms and alliances provide military capability, and military spending has domestic opportunity costs, states can rely on allied military spending instead of their own. 
The public goods theory of alliances also predicts that alliance membership leads to reduced military spending, but uses a different process. 



\subsection{Public Goods Theory of Alliances} 

% Introduction to the public goods theory
Also known as the economic theory of alliances, public goods theory starts with the premise that security from an alliances is a collective good for members \citep{OlsonZeckhauser1966}.
Collective or public goods have two key characteristics, non-excludability and non-rivalry. 
Consumption of alliance security is non-excludable--- denying another alliance member protection from the pact undermines the purpose of the treaty.  
Because alliance security is non-rival, consumption of security by one member does not reduce the security available to other members. 


% Collective action problem
Public goods theory treats each alliance member's military spending as a contribution to collective security. 
Alliances aggregate the capability of their members, so the total capability of an alliance depends on members' military expenditures. 
Through investment in arms, alliance members provide the organization with additional capability and increase the amount of security it provides. 
But because alliance security is a public good, its provision is subject to a collective action problem. 


In collective action problems, individual contributions provide a benefit for all, but individual benefits are smaller, which creates an incentive to rely on others to provide the public good.
Self-interested individuals will contribute less than needed, leading to inadequate provision of the public good. 
In alliances, individual members depend on the other states to provide military capability. 
By spending less, alliance members can consume more non-defense goods, but the alliance provides less security.


% Now we're free, free riding.  
Relying on others' contributions to a public good is called free-riding.
Free-riding in alliances, is reflect in allocations of resources to the military.  
\citet{OlsonZeckhauser1966} focus on the defense burden--- military expenditures as a share of GDP. 
Given the collective action problem, they expect that larger members of the alliance will bear a higher defense burden, because they value security from the alliance more. 
If this logic holds, economic size and defense burdens will be positively correlated--- larger alliance members will spend a higher share of their resources on defense. 


% Joint product model extension
An extension of the public goods theory of alliances argues that military spending produces public and private goods for states \citep{Hansenetal1990, Murdoch1995}. 
In this joint product model, some military spending provides for collective security through alliances, which is the public good. 
Other spending provides private goods such as internal control, border security, and particular deterrence. 
The joint product model expects that as states reap more private benefits from military spending, their defense burden will increase \citep{SandlerHartley2001}, and alliance members will only free-ride in military spending for collective security. 



\subsubsection{Evidence}


% Prediction and measurement
The public goods theory of alliances predicts that among alliance members, larger states will have a higher defense burden. 
The key independent variable is economic size, measured using Gross Domestic Product (GDP) or Gross National Product (GNP). 
Defense spending as a share of GDP or GNP is the dependent variable. 


% Evidence
Because \citet{OlsonZeckhauser1966} first tested their claims of free-riding on the North Atlantic Treaty Organization (NATO), and NATO members have quality data, many tests of the public goods theory of alliances only examine NATO.\footnote{In the interest of parsimony, I include only key works on free-riding in NATO.}
Other scholars have examined alliances besides NATO, where some find evidence of disproportionate military expenditures, while others do not. 
\autoref{tab:free-ride-sum} details the results of previous studies of free-riding in alliances. 


\begin{table}[hbt]
\begin{tabular}{lcc}
  & Free-Riding & Alliance Scope \\
\hline
\citet{OlsonZeckhauser1966} & Yes  & NATO 1964 \\
\citet{Starr1974} & Yes & Warsaw Pact 1967-1971 \\
\citet{Reisinger1983} & No & Warsaw Pact 1970-1978 \\
\citet{Thies1987} & Mixed & Seven pre-1945 Alliances \\ 
\citet{ConybeareSandler1990} & No & Triple Alliance \& Triple Entente 1880-1914 \\
\citet{Palmer1990} & No & NATO 1950-1978 \\
\citet{Chenetal1996} & No & Arab League 1950-1988 \\
\citet{OnealWhatley1996} & Yes & NATO, Rio Pact \& Arab League 1953-1988 \\
\citet{Siroky2012} & No & Quintuple Alliance Members 1820 \\
\citet{PluemperNeumayer2015} & Yes & NATO 1956-1988 \\
\hline 
\end{tabular}
\caption{Findings on Free-Riding in Alliances. Most of these studies use military expenditures as a share of GDP for the dependent variable, and assess whether it is correlated with economic size. A positive correlation between GDP or GNP and the defense burden among alliance members is the standard evidence of free-riding.}
\label{tab:free-ride-sum}
\end{table}


There is mixed evidence of free-riding across these ten studies. 
While most studies of NATO find the expected positive correlation between GDP and defense burdens, the evidence is less consistent in other contexts. 
Free-riding is absent among front-line Arab League members, the Triple Alliance, Triple Entente, and the Quintuple Alliance. 
Two studies of the Warsaw Pact reach different conclusions. 
Differences in results across studies could reflect contextual differences or problems with estimating correlations between GDP and the defense burden. 


% Problems with ratio DV
Interpreting regression models with a ratio dependent variable is difficult, because it is impossible to identify whether changes in the numerator or denominator are driving the results. 
GDP is part of the dependent and independent variables, which further complicates interpretation of the coefficients in a regression model of free-riding.
Larger states also have broad foreign policy interests that include multiple alliances, so their defense burden reflects more than their contribution to a single alliance.  


\citet{PluemperNeumayer2015} avoid these identification problems and provide the best evidence of free-riding in NATO by estimating a quasi-spatial model of how growth in NATO members military spending responds to US and Soviet spending. 
In this model, a lack of responsiveness to increasing Soviet spending implies free-riding on US protection. 
They find that many NATO members did not increase their military spending when Soviet spending exceeded US spending, and that the extent of free-riding depends on proximity to the Warsaw Pact.
Crucially, their estimates of the degree of free riding are uncorrelated with GDP, which contradicts Olson and Zeckhauser's expectation that smaller states are more likely to free ride. 


% final verdict
Due to their obsession with free riding, public goods theories of alliances have added less to our understanding of how alliance participation affects military spending. 
The main observable implication of public goods theory creates serious identification problems in research design, and is not supported in the best-identified model \citep{PluemperNeumayer2015}. 
Moreover, the concept of free-riding itself is problematic in the context of alliances. 


Alliance treaties reflect exchanges or bargains among members, and one possible bargain is a security-autonomy tradeoff, where smaller members give up foreign policy freedom in exchange for protection \citep{Morrow1991}. 
For example, European defense spending has remained low in part because the US made its protection conditional on European states not pursuing independent nuclear capabilities \citep{Lanoszka2015}. 
These shortcomings have important consequences for the prediction that alliance membership reduces military spending, which I assess in the next section. 



\subsection{What do we know about reduced military spending by alliance participants?}


% Introduce aggregate assessment
Of the two models that predict alliance membership reduces military spending, substitution theory of foreign policy contains more useful insights. 
By treating security as a private good for states, substitution theories of foreign policy avoided debates over free-riding and burden sharing. 
Mixed results from early tests, coupled with the theory and results of \citet{DigiuseppePoast2016} suggest that alliance membership reduces military spending in particular circumstances. 
Alliance participation may not always reduce military spending because arms and alliances are imperfect replacements. 


% Imperfect substitutes
Military spending and alliances provide capability in different ways, so the policies differ in their speed and reliability.
States can rely on their own arms in any situation, but domestic military capabilities take a long time to develop. 
Because alliance members have divergent foreign policy interests and do not always fulfill their commitments,\footnote{Recent estimates suggest that about 50\% of alliance commitments are honored \citep{BerkemeierFuhrmann2018}.} alliances are a less reliable source of capability than domestic arms, but provide immediate capability gains. 
When a treaty enters into force, alliance members gain their partner’s support with some probability.


Imperfect substitution between arms and alliances has important consequences for our models of how alliance membership affects military spending. 
If a state believes its allies will honor their treaty obligations, then that treaty is more like domestic military arms--- a highly reliable source of capability.
\citet{DigiuseppePoast2016} use differences in reliability to justify their prediction that defense pacts with democratic states will lead reduced defense expenditures, as democracies are more reliable alliance partners than other states.


The other component of imperfect substitution is how quickly states develop domestic military capability.  
The faster states can convert military expenditures into capability by training troops and building weapons, the better domestic arms substitute for immediate capability gains from an alliance. 
The implications of speed and cost in converting defense investments into military capability for alliance participation have not been addressed, and examining this question would add further evidence to our assessment of the substitution theory of foreign policy. 


% Different costs, especially FP autonomy in alliances 
Another unexplored aspect of imperfect substitution is differences between the costs of arms and alliances. 
States bear different costs when they use arms or alliances to build capability.
Military spending employs financial resources and human capital. 
Alliances impose limit members' freedom of action in foreign policy. 


% Use imperfect substitutes to transition to complements section 
Imperfect substitution of arms for alliances can also motivate a different prediction of how alliance membership affects military spending.
Scholars have used the other characteristics of arms and alliances to argue that alliance participation could increase military spending.
Others note that the way states use alliances and arms will generate a positive correlation between the two. 
In the next section, I address the predictions that alliance membership leads to increased defense spending. 




%----------------------------------
\section{Why Alliance Participation Increases Military Spending}


% Intro para
If alliance participation increases military spending. 
This prediction contradicts unconditional expectations from the substitution theory of foreign policy and public goods theory of alliances. 
Most authors in this literature frame their work as a riposte to predictions that alliance participation reduces defense spending. 


% Why does this prediction matter? 
If alliance participation can increase military spending, then the common perception that alliances lead to reduced spending or free-riding should be revised. 
Due to the salience of debates over burden-sharing in NATO, the prediction of decreased spending is better known. 
Predictions of increased spending emphasize differences between arms and alliances. 


% No shared mechanism, but some shared ideas 
Substitution models emphasize domestic political incentives to use alliances as a substitute for military spending.  
Most opposing models argue that alliances expand what a state can achieve in international relations. 
Four models share this focus on foreign policy gains from alliance membership. 



\subsection{Models and Evidence} 


% snyopsis of how models have changed.
Most models in this group acknowledge that alliance participation decreases military expenditures.
Then these same models argue that in many circumstances, increases are more likely. 
Unfortunately, it is hard to compare results from tests of these theories with other work due to research design problems. 
Still, the models contain some useful theoretical insights about the foreign policy value of alliances. 


\subsubsection{Diehl 1994}


% Weaknesses of perfect substitution models
\citet{Diehl1994} argues that arms and alliances are imperfect substitutes, so alliance participation can lead to increased military spending. 
Each policy has different motives besides external threat, which weakens the extent of substitution between arms and alliances.
Substitution is further weakened by the ability of states to use other policies like arms imports or negotiation in response to an external threat. 


% Two paths 
Although Diehl argues that reductions in military spending are smaller than other models of the arms-alliances tradeoff predict, he acknowledges it is possible.
The contribution of the paper is describing two mechanisms by which alliance participation leads to increased military spending. 
First, joining an alliance can increase a state's foreign policy responsibilities. 
After forming a treaty, states may deploy troops abroad, spend more on particular weapons to meet threats from their ally's other enemies, or become involved in conflict, all of which will increase military spending. 
States also form alliances to support other states, and that support requires additional spending. 

The second mechanism is related to the first, and presumes the existence of free-riding. 
If alliance participation leads to reduced military spending by smaller states, larger partners will need to increase spending to ensure the alliance continues to provide adequate security. 
As larger states form alliances to increase their autonomy, or freedom of action in foreign policy \citep{Morrow1991}, the price of this change may be increased spending to ensure the security of client states. 
As a result, alliance participation is especially likely to increase military spending in larger states as they pick up the slack of their smaller partners. 


% Evidence 
To test his predictions, Diehl examines a set of major power rivalries from 1816 to 1976, because there is a salient external threat and major powers can use alliances as substitutes or complements for arms.
The unit of analysis is the rivalry year, and the change in each rival state's military spending is the dependent variable. 
Because major powers are part of multiple rivalries at a time, many annual great-power observations are repeated in this dataset. 
For example, there are three observations of changes in Japan's military spending in 1937, due to Japan's participation in rivalries with the UK, US, and USSR in that year. % They were opposed to united anything at the time, I guess. 
Diehl's t-tests and regression models treat these repeated observations as independent of one another, which is untenable. 


% Final assessment of Diehl. 
Furthermore, the sample of major power rivalries selects cases based on the dependent variable, which is likely to result in biased coefficient estimates. 
Therefore, Diehl's reported results are not trustworthy. 
The focus on major-power rivalry years also makes it difficult to compare these results with results from state-year data. 
Despite research design shortcomings, Diehl's insight that alliances can increase a state's foreign policy obligations is useful.  
\citet{MorganPalmer2006}'s work has similar characteristics--- an interesting theoretical insight and poor research design. 



\subsubsection{Morgan and Palmer 2006}


% Intro 
Morgan and Palmer's expectations about the relationship between arms and alliances are part of a general model of foreign policy. 
Due to the scope of the theory, it is necessary to summarize the whole model. 
After describing their ``two-good'' model of foreign policy, I turn to the mechanism behind their prediction that alliance participation increases military spending. 


% Baseline assumptions 
The first premise of this model is that states formulate foreign policy with reference to the status quo in international politics. 
The status quo includes multiple issue areas, where states can seek maintenance or change. 
Maintenance tries to keep favorable parts of the status quo on an issue.
Change seeks to alter unfavorable aspects of the status quo.
All states have a mix of maintenance and change objectives. 


% Foreign policy budget/expenses 
To maximize their satisfaction from foreign policy, states expend all available foreign policy resources to produce a mix of maintenance and change. 
Spending the entire foreign policy budget leads to a tradeoff between the two goals--- more maintenance in one area requires less resources for change in another. 
This tradeoff between maintenance and change corresponds to the two-good utility maximization model, where states spending their entire budget to maximize consumption of two goods.


% Allocation of resources across policies
In international relations, states produce maintenance and change by allocating resources to foreign policy instruments such as military spending, alliances, foreign aid, or negotiations. 
Some policies produce more maintenance, while others produce more change. 
Shifts in the overall foreign policy budget are the primary cause of changes in allocation of resources to particular policies \citep{MorganPalmer2000}.


% Increased spending
An increase in the resources a state can spend on foreign policy will lead to an increase in resources for most policies.
Absent a change in state preferences, the relative importance of each policy in the total foreign policy portfolio will remain the same. 
Given additional resources to spend on foreign policy, Morgan and Palmer expect that states will increase their use of multiple policies simultaneously. 


% Application to arms and alliances
How does the above model apply to alliance membership and military spending? 
Alliances increase a state's foreign policy budget and provide more resources to pursue foreign policy goals.
By expanding what a state can achieve in international politics, alliances allow states to devote more resources to other policies, like military spending \citep{MorganPalmer2003}. 
As an illustrative example, Morgan and Palmer argue that an alliance with the US allowed Britain and France to pursue change in the Middle East during the 1957 Suez crisis.
Without US protection, Britain and France would have needed to spend more protecting themselves from the Soviet Union, but instead they used their military resources in an attempt to reverse the nationalization of the Suez Canal.


The crux of Morgan and Palmer's argument is that ``if an alliance enables a state to produce its desired foreign policy goods with fewer resources than it could otherwise, this actually frees resources for the pursuit of additional goals.''
This income effect, where alliance members can achieve more in foreign policy, is the key mechanism behind Morgan and Palmer's expectation that alliance membership leads to increased military spending. 
As a final step, Morgan and Palmer explain why increases in military spending are more likely than decreases for alliance members. 


% Substitution
Morgan and Palmer acknowledge states can use multiple policy instruments to realize a particular goal. 
To incorporate substitution into their model, they assume that states employ the most efficient policy for a particular goal, given their resource constraints. 
States use the policy that gets them the most maintenance or change at the lowest cost.


Because foreign policy resources are already allocated optimally, substitution of one policy instrument for another is rare in this model. 
Morgan and Palmer only expect states to transfer resources from one policy to another if the relative efficiency of those policies has changed. 
Security-seeking states would spend less on domestic arms and rely on alliances only if alliances became a more efficient provider of security relative to arms. 
They then assert that changes in the relative efficiency of policies are rare, and changes in resources are common.
Therefore, alliance members will usually increase military spending. 


% Research design
Morgan and Palmer test their expectation that on average alliances lead to increased military spending using a generalized least squares regression model.\footnote{The book tests other predictions of their theory, but I focus on Table 3 in \citet{MorganPalmer2003}, which is also Table A.4 in \citet{MorganPalmer2006}.}
The dependent variable is changes in military spending.
The key independent variables are a state's power, the power gained from an alliance, and their interaction, so interpretation focuses on marginal effects.\footnote{State size explains relative emphasis on change or maintenance in the overall model, so levels and changes are part of their predictions.} 
Power is measured using CINC scores, which capture a state's share of material capabilities in the international system, and alliance power is the weighted sum of allied states' power. 

Morgan and Palmer find that the marginal effect of joining an alliance on military spending is positive for all states. 
Unfortunately, they do not report uncertainty estimates for these marginal effects, so their comparisons of the effects of different alliances are not meaningful \citep{BramborClarkGolder2006}. 
The estimated marginal effects are also based on a regression model with no control variables.
Because the model omits variables that are correlated with military spending and alliance behavior, such as threat level and international conflict involvement, the coefficient estimates are biased. 


% Sample is a problem
Furthermore, Morgan and Palmer test their expectations in a problematic sample.
They focus on all states whose alliance portfolio changed from 1816 to 1992, and measure changes in military spending for four years after the portfolio changes took effect.
The use of four-year period after a change in an alliance portfolio is arbitrary. 
The estimation sample includes 858 observations, which implies 214 instances of alliance portfolio change.  


From 1816 to 1992 there are 367 alliances in the Correlates of War project's alliance data.
The formation of each of these alliances alone changes the portfolio of each member. 
If all alliances had two members, and Morgan and Palmer measured 4 years after the formation of each, there would be 2,936 observations in the sample. 


If formation, dissolution and changing membership of alliances all change alliance portfolios, there should be thousands of observations in Morgan and Palmer's sample.\footnote{This applies even if most alliances involve major powers. NATO alone could produce 200 observations, and formation of the Rio Pact should generate 92 observations.} 
Therefore, the number of observations in Morgan and Palmer's sample is far lower than what their description of the research design suggests.
This disparity between the stated research design and the estimation sample reduces the face validity of the results and raises the risk of sample selection bias.\footnote{There is no publicly available replication data, and reconstructing their data would be time-consuming.} 


% final assessment
Due to research design flaws, Morgan and Palmer's estimates of the relationship between alliance participation and military spending are invalid. 
The theoretical insights that alliances expand what a state can achieve in international politics and states pursue both maintenance and change are helpful, however. 
The next model argues that states use increased defense effort to make themselves a more attractive alliance partner. 



\subsubsection{Horowitz et al 2017} 

\citet{Horowitzetal2017} examine how domestic politics affect international cooperation. 
The particular focus is the relationship between implementing conscription and alliance formation. 
Using conscription in place of a volunteer military requires coercing citizens to serve, and is unpopular. 


By switching to conscription, states signal to prospective alliance partners that they will not abandon them or free-ride. 
Conscription is a credible signal of commitment to a robust defense posture, because states bear domestic political costs.
Large conscript armies are valuable in war, so conscription signals that a state can fight well if needed. 
Therefore, a switch to conscription is a means for states to signal they are a worthy alliance partner. 


Horowitz, Poast and Stam predict that states which implement a switch to conscription will be more likely to form alliances. 
In this model, investment in domestic arms increase alliance participation, so arms and alliances are complements. 
Through sacrifice in domestic politics, states can realize foreign policy gains from an alliance. 


% Evidence
To assess their prediction that a switch to conscription increases the probability of alliance formation, Horowitz, Poast and Stam examine alliance formation in a sample of states from 1815 to 2001. 
The dependent variable is a binary indicator of whether a state joined a defensive alliance in a given year. 
The key independent variable is a binary indicator of whether a state switched from an all-volunteer military to conscription in the previous 5 years. 
Using logit models that control for threat, defense burden, and other possible omitted factors, they find the expected positive correlation between a switch to conscription and alliance formation. 
A switch to conscription increases the probability of joining a defensive alliance by 150 percent. 


% Final assessment 
Horowitz, Poast and Stam's insight that states can use domestic arms as a signal to potential alliance partners is a novel and plausible explanation of a positive association between defense effort and alliances. 
Their results are robust to multiple model specifications, including interactions with the international threat environment. 
In sum, this paper provides another theoretical path to a positive association between alliance participation and military spending, and establishes the best empirical evidence of a that association to date. 
The last argument linking alliances to increases in defense effort believes that alliances facilitate cooperation among members.  



\subsubsection{Cooperation in Alliances} 


% Palmer 1990: Iterated PD as a metaphor, because actors are more sophisticated. 
\citet{Palmer1990} argues that models of free-riding use the wrong bargaining game. 
Deterrence from an alliance is a public good. 
Instead of a Cournot bargaining game with short-term payoffs, Palmer argues that an iterated prisoner's dilemma is a better model. 
Prisoner's dilemma models are a common framework for studying public goods \citep{Axelrod1984}.


In a prisoner's dilemma, the actors get more benefits when they cooperate than when they all defect. 
But if a single actor defects while the others cooperate, they get the best possible payoff.
Therefore, every individual has an incentive to defect. 
When the prisoner's dilemma has one round, individual incentives lead all players to defect and everyone is worse off. 
Actors in a standard prisoner's dilemma model cannot communicate or use side-payments, but Palmer argues that alliance members can engage in these sophisticated strategies to ensure cooperation. 


Palmer uses the prisoner's dilemma as a metaphor for bargaining among alliance members over defense spending. 
Unlike in the standard prisoner's dilemma, states can communicate and use side-issues during bargaining over defense spending in an alliance.
Communication allows states to coordinate to ensure an adequate supply of deterrence. 
Payments on side-issues make defection more costly. 
Sophisticate bargaining by members allows an alliance to provide more deterrence \cite{Hardin1982}.\footnote{Suboptimal provision of deterrence by the alliance is reduced, not eliminated.} 


% Evdidence
Because alliance members can communicate and use side-payments, Palmer expects less free-riding among alliance members than the public goods theory of alliances.  
Like many studies of the public goods theory of alliances, Palmer tests his predictions on a sample of Western European states from 1950 to 1978. 
He finds that increases in the US defense burden are associated with higher defense burdens for large and small members of NATO.\footnote{Defense burden in this case is military expenditures as a share of GNP}.  


% The other test of the cooperation logic 
\citet{QuirozFlores2011} offers a different test for cooperation among alliance members, where shared alliance membership establishes contiguity in a spatial autoregressive model.
Contiguity implies an association between the defense burdens of allied states. 
In a sample of 119 states in 2000, Quiroz-Flores finds a positive spatial correlation between defense burdens in allied states. 
He then asserts that alliance members respond to increased allied spending by increasing their own spending, but his cross-sectional data do not test this dynamic story.



% No evidence of the mechanism in action here. 
\citet{Palmer1990} and \citet{QuirozFlores2011} use defense burden as their dependent variable.
Therefore, their models suffer the same identification problem as tests of the public goods theory of alliances. 
Also, there is no evidence of the theoretical mechanism in action. 

If states use side payments to induce cooperation, we can observe those payments. 
Looking for evidence that allied states use side payments to boost defense effort would be a better test of the cooperation model. 
This implies a distinct research question--- can side payments facilitate burden-sharing in defense? 


% Does communication work?
Although alliance members can communicate to try and enforce norms of equitable burden-sharing, communication may not work.  
Every US administration since Eisenhower has bemoaned free-riding by NATO members to little effect \citep{Lanoszka2015}. 
Communication about burden-sharing does not necessarily override the domestic incentives to reduce defense spending. 



% Summarize cooperation and transition
The cooperation model of arms and alliance calls attention to the bargaining process among alliance members. 
It is not clear that side payments or communication among alliance members generate increased defense spending, but the side-payments mechanism cannot be ruled out yet. 
Problems with the cooperation model reflects other models of a positive association between arms and alliances.  



\subsection{What do we know about increased military spending by alliance participants?} 


% Start with key insight- alliances produce foreign policy gains 
There are four distinct models of increased military spending by alliance members.
All four focus on the foreign policy role of alliances. 
Assessing the relative merits of each model has been hindered by research design problems. 


% Note research design flaws
Only one of the four models also has a robust empirical test. 
Tests of the other three theories suffer from a mix of odd samples and model misspecification. 
Thus, there is not much evidence for or against a positive relationship between alliance participation and military spending--- though the results of \citet{Horowitzetal2017} suggest it is possible. 


% Plausible, but less clear. Rather than focus on one model, focus on general idea of FP role of alliances 
The premise that alliances increase engagement in international politics is reasonable. 
None of the above models address domestic incentives to reduce military spending, however. 
We can better model any positive association between alliance participation and military spending by integrating insights from the substitution theory of foreign policy. 




%----------------------------------
\section{Conclusion}


% Introduce gaps and associated research questions: make sure it's clear where this section is going
Combining competing models and their predictions is one of several gaps in research on the relationship between alliance participation and military expenditures. 
This final section describes the state of knowledge on the arms-alliances tradeoff and the gaps in that knowledge. 
Prior scholarship has made some important progress, but there is ample room for additional work. 


% State of knowledge
The substitution theory of foreign policy is solid, but substitution of alliances for arms probably depends on alliance characteristics. 
Thanks to research design problems, the extent of increased defense spending by alliance members is uncertain at the moment, but the results of \citet{Horowitzetal2017} imply it is possible. 
Complementarity theories contain the kernel of an important idea--- alliances may lead states to increase arms because they produce foreign policy gains. 


% focus on characteristics
So far, debate between the two camps has focused on the characteristics of arms and alliances. 
Predictions of decreased military spending by alliance members emphasize how both policies provide security. 
Competing predictions focus on other characteristics, especially the role of alliances as an international institution. 


% Different underlying models of conflict.
An unstated aspect of the division in the literature is reliance of different models of international conflict. 
Predictions that alliance membership reduces defense spending employ a deterrent model of conflict. 
In a deterrent model, alliances provide capability to offset a threat, allowing members to reduce spending.
The prediction that alliance participants will spend more of the military relies on a spiral model of conflict.
In a spiral model, alliances are part of a cycle where disputes continue to escalate and military spending is also increasing. 
There is substantial debate in the literature on conflict onset between the spiral and deterrent models \cite{Jervis1978, Reiter1995, SeneseVasquez2008, JacksonMorelli2008, JohnsonLeeds2011, Kenwicketal2015}


% Gaps
There are three gaps in our understanding of how alliance participation impacts military spending, and each gap generates a research question I plan to address. 
After articulating each research question, I sketch a possible answer. 
The first gap is the disconnect between models from the two camps. 



\subsection{Combining Predictions} % Need to find a better label for this. 


% General summary
Models of decreased defense spending in alliances emphasize domestic political incentives to reduce military spending. 
Models of increased spending focus on the foreign policy role of alliances. 
Neither perspective incorporates the key insight of the other.


% Gap: Association between the two theories 
As a result, the two theoretical camps are disconnected from one another. 
For example, \citet{Horowitzetal2017}'s argument starts with the premise that free-riding occurs, but does not clarify when states free-ride and when they use increased defense effort to participate in an alliance. 
\citet{DigiuseppePoast2016} use mixed findings in previous work to motivate their paper, but only predict decreases in spending. 


Division between the two sets of models is the biggest research gap. 
Currently, scholars treat predictions of increased and decreased spending as incompatible. 
But both predictions are plausible for different reasons. 
States can use alliances for security to increase domestic consumption, or foreign policy gains. 
But we have not yet placed the two mechanisms in a unified framework. 


% Introduce the question and the answer. 
This theoretical omission raises the first research question--- when do alliances substitute for or complement domestic arms? 
To combine these models, researchers should clarify the process behind how alliances impact military spending. 
Some models use an indirect mechanism--- alliances change the risk of conflict, and members change military spending accordingly. 
In contrast, \citet{Diehl1994} and \citet{MorganPalmer2006} argue that alliance participation directly increases the marginal utility of military spending.


% argue for the indirect model
The indirect conflict risk model provides a more intuitive means of predicting both increases and decreases in spending.
Whether alliances increase or decrease the probability of international conflict is hotly debated \citep{SiversonSullivan1984, Smith1995, Colaresi2005, JohnsonLeeds2011, Kenwicketal2015, Kang2017}.
Different alliances have heterogeneous effects on conflict, which so these treaties can also have different effects on military spending. 


Using two classifications of alliances, \citet{Benson2011}, \citet{Leeds2003}, and \citet{JohnsonLeeds2011} find that participation in some alliances increases the risk of conflict, while others reduce it. 
A model of the arms-alliances tradeoff could build on these results.\footnote{Building on these results requires robustness checks.} 
Alliances that increase the risk of conflict lead to increased investment domestic arms, while alliances that decrease conflict risk allow states to reduce military expenditures. 
Considering the role of conflict risk in the arms-alliances tradeoff also translates well into other research questions, such as the role of the international context. 



\subsection{International Context and the Arms-Allies Tradeoff}


% Gap: International Context: nature of threat changing over time due to implications of conflict. Reflected in pooling assumption of observations, or imposition of fixed coefs. 
Neither theory nor research design addresses how changes in the international system affect the relationship between arms and alliances. 
Domestic characteristics and international threat affect military spending, but the role of the international context is not well understood.
Systemic factors alter the costs and benefits of international conflict.
Between the number of major powers, nuclear weapons, changing norms about international conquest \citep{Fazal2011}, and economic interdependence \citep{Frieden2006}, the international environment has changed in dramatic ways.


After 1945, all of these shifts constrained international conflict. 
Most of the evidence that alliance participation reduces military spending comes from the post-World War II period, however.  
The added security of an alliance in a time with low conflict risk may have allowed states to dramatically reduce military spending. 


If the association between alliance participation and arms is altered by systemic factors, current empirical tests do not capture it. 
Many studies pool observations over from 1816 to the present, and none allow coefficients to vary over time. 
At best, \citet{Horowitzetal2017} control for the post-45 period with dummy variable, but the international context may modify the relationship between arms and alliances. 


% Research Question
How might the international context alter the arms-alliances tradeoff?
The consequences of international conflict depends in part on international factors. 
While some models of conflict assume that losing a war leads to lost sovereignty \citep{Fearon2018}, only two states have ``died'' as the result of losing a war since 1945 \citep{Fazal2011}. 
The downside risks of conflict changes over time. 


The consequences of conflict matter because alliances are not perfectly reliable. 
\citet{BerkemeierFuhrmann2018} find that half of all alliances are honored. 
But if states can lose a war and survive, they may be more inclined to substitute alliances for arms. 


The international context may alter the magnitude of any association between alliance participation and military spending. 
By this logic, we would expect rampant substitution in the post-World War II period, and less beforehand. 
If this is true, it has important consequences for the final research gap--- our lack of systematic empirical knowledge. 



\subsection{Extent of Substitution and Complementarity}


% Gap: Research design challenges. We don't know much about the extent of substitution or complementarity. 
We do not know how many alliances substitute for or complement domestic arms.
Though the prevalence of complementarity and substitution is the central debate, there are no studies that compare the effects of participation in many different alliances. 
Current research designs are divided between specific and general studies of alliance participation and military spending.
Neither design describes the extent of increases and decreases in spending across alliances.  


% Specific vs general studies of alliance membership and arms 
% Specific
Specific studies of alliance participation and military spending focus on a few treaties. 
Usually these studies estimate the response of alliance members to changes in military spending by a key alliance partner. 
Though specific studies provide solid evidence in particular cases, their results may not apply to other alliances. 
Studies of specific alliances provide a few data points with which to assess how many alliances have each association, but this approach is not comprehensive. 


% General
General studies compare states based on differences in their alliance participation. 
The standard independent variable in these models is a dummy indicator of alliance membership which compares states with some alliance to states without one. 
Binary indicators of alliance membership collapse alliance-level variation down to the state level. 
This approach produces more generalizable patterns, but does not compare alliances.  


% Sum it up 
Because specific studies focus on a few alliances, and general studies compare states, we do not know how many alliances increase military spending, and how many reduce it.
This is a simple but important gap in our knowledge. 
It is primarily an empirical question. 


% Research Question
How many alliances lead members to decrease military spending? 
How many alliances lead members to increase military spending? 
It is possible to address these two questions in a unified statistical framework. 


The shortcomings of specific and general studies reflect a levels of analysis problem. 
As international organizations, alliances are a distinct level of analysis from states \citep{Mattes2012}. 
Conditional theories of the arms-alliances tradeoff emphasize differences between alliances which do not connect directly to military spending at the state level. 
A statistical approach that preserves the full range of state and alliance-level variation would compromise between specific and general studies. 


% ML models as a possible solution
Multilevel modeling is one solution to the levels of analysis problem \citep{GelmanHill2007, McElreath2016}. 
By connecting the state and alliance levels of analysis, multilevel models can estimate general associations between alliance characteristics and military spending and the aggregate impact of individual alliances.
Partial pooling of observations in a multilevel model, would generate plausible estimates for short-lived treaties with few members by borrowing strength from alliances with more information. 
Multilevel models can provide a measure of substitution and complementarity for each alliance. 


% Caveat emptor
The exact model used to answer this third gap depends on theoretical answers to the first two questions. 
Absent theoretical guidance for research design, there is a high risk of spurious inferences. 
We can only estimate the extent of complementarity and substitution given predictions about which types of alliances are associated with each effect. 


% Transition to the end. 
Addressing this third gap in the literature would provide novel evidence of the relationship between arms and alliances. 
Additional evidence would help assess past theories and develop new ones.
Estimating the association of individual alliances with members' military spending will also raise new questions, and stimulate further interest in this area of research. 



\subsection{Final Thoughts} 

% Concluding paragraph. 
Whether alliance participation increases or decreases military spending is worthy of additional scholarly attention. 
If alliances only substitute for domestic arms, policymakers will struggle to coerce their allies to spend more without weakening the alliance. 
Further scholarship can also address the split between two competing perspectives in previous work.


The substitution theory of foreign policy and public goods theory of alliances predict that alliance participation reduces military spending. 
Another diverse set of models expect that alliance participation increases military spending. 
Neither camp has a monopoly on the truth. 
Bridging these perspectives is the key task for future scholarship, and it would clarify an important issue in international relations. 









\section*{Appendix}

\subsection*{Visual Summary of Prior Results} 


\begin{table}[hbt!]
\begin{tabular}{lccc}
     & Decrease & Increase & Null \\
\hline
\citet{MostSiverson1987} &  &  & X \\
\citet{Morrow1993} & X &  &  \\ 
\citet{Conybeare1994} & X & &  \\
\citet{Diehl1994} &  & X &  \\
\citet{Goldsmith2003} &  &  & X \\
\citet{MorganPalmer2006} &  & X & \\ 
\citet{QuirozFlores2011} &  & X &  \\ 
\citet{DigiuseppePoast2016} & X &  & \\ 
\citet{Horowitzetal2017} &  & X & \\ 
\hline
\end{tabular}
\caption{General Findings of Association Between Alliance Membership and Military Spending}
\end{table}





\singlespace


\bibliography{C:/Users/jkalley14/Dropbox/Research/MasterBibliography}  
%\bibliography{C:/Users/Josh/Dropbox/Research/MasterBibliography} 





\end{document}